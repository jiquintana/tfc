%
% API Documentation for API Documentation
% Module ThreadPool
%
% Generated by epydoc 3.0.1
% [Thu Nov 28 12:12:35 2013]
%

%%%%%%%%%%%%%%%%%%%%%%%%%%%%%%%%%%%%%%%%%%%%%%%%%%%%%%%%%%%%%%%%%%%%%%%%%%%
%%                          Module Description                           %%
%%%%%%%%%%%%%%%%%%%%%%%%%%%%%%%%%%%%%%%%%%%%%%%%%%%%%%%%%%%%%%%%%%%%%%%%%%%

    \index{ThreadPool \textit{(module)}|(}
\section{Module ThreadPool}

    \label{ThreadPool}

%%%%%%%%%%%%%%%%%%%%%%%%%%%%%%%%%%%%%%%%%%%%%%%%%%%%%%%%%%%%%%%%%%%%%%%%%%%
%%                               Variables                               %%
%%%%%%%%%%%%%%%%%%%%%%%%%%%%%%%%%%%%%%%%%%%%%%%%%%%%%%%%%%%%%%%%%%%%%%%%%%%

  \subsection{Variables}

    \vspace{-1cm}
\hspace{\varindent}\begin{longtable}{|p{\varnamewidth}|p{\vardescrwidth}|l}
\cline{1-2}
\cline{1-2} \centering \textbf{Name} & \centering \textbf{Description}& \\
\cline{1-2}
\endhead\cline{1-2}\multicolumn{3}{r}{\small\textit{continued on next page}}\\\endfoot\cline{1-2}
\endlastfoot\raggedright p\-y\-t\-h\-o\-n\-\_\-O\-l\-d\-V\-e\-r\-s\-i\-o\-n\- & \raggedright \textbf{Value:} 
{\tt True}&\\
\cline{1-2}
\raggedright \_\-\_\-p\-a\-c\-k\-a\-g\-e\-\_\-\_\- & \raggedright \textbf{Value:} 
{\tt None}&\\
\cline{1-2}
\end{longtable}


%%%%%%%%%%%%%%%%%%%%%%%%%%%%%%%%%%%%%%%%%%%%%%%%%%%%%%%%%%%%%%%%%%%%%%%%%%%
%%                           Class Description                           %%
%%%%%%%%%%%%%%%%%%%%%%%%%%%%%%%%%%%%%%%%%%%%%%%%%%%%%%%%%%%%%%%%%%%%%%%%%%%

    \index{ThreadPool \textit{(module)}!ThreadPool.ThreadPoolMixIn \textit{(class)}|(}
\subsection{Class ThreadPoolMixIn}

    \label{ThreadPool:ThreadPoolMixIn}
\begin{tabular}{cccccc}
% Line for SocketServer.ThreadingMixIn, linespec=[False]
\multicolumn{2}{r}{\settowidth{\BCL}{SocketServer.ThreadingMixIn}\multirow{2}{\BCL}{SocketServer.ThreadingMixIn}}
&&
  \\\cline{3-3}
  &&\multicolumn{1}{c|}{}
&&
  \\
&&\multicolumn{2}{l}{\textbf{ThreadPool.ThreadPoolMixIn}}
\end{tabular}

\textbf{Known Subclasses:} main.ThreadedServer

use a thread pool instead of a new thread on every request


%%%%%%%%%%%%%%%%%%%%%%%%%%%%%%%%%%%%%%%%%%%%%%%%%%%%%%%%%%%%%%%%%%%%%%%%%%%
%%                                Methods                                %%
%%%%%%%%%%%%%%%%%%%%%%%%%%%%%%%%%%%%%%%%%%%%%%%%%%%%%%%%%%%%%%%%%%%%%%%%%%%

  \subsubsection{Methods}

    \label{ThreadPool:ThreadPoolMixIn:keep_running}
    \index{ThreadPool \textit{(module)}!ThreadPool.ThreadPoolMixIn \textit{(class)}!ThreadPool.ThreadPoolMixIn.keep\_running \textit{(method)}}

    \vspace{0.5ex}

\hspace{.8\funcindent}\begin{boxedminipage}{\funcwidth}

    \raggedright \textbf{keep\_running}(\textit{self})

\setlength{\parskip}{2ex}
\setlength{\parskip}{1ex}
    \end{boxedminipage}

    \label{ThreadPool:ThreadPoolMixIn:force_shutdown}
    \index{ThreadPool \textit{(module)}!ThreadPool.ThreadPoolMixIn \textit{(class)}!ThreadPool.ThreadPoolMixIn.force\_shutdown \textit{(method)}}

    \vspace{0.5ex}

\hspace{.8\funcindent}\begin{boxedminipage}{\funcwidth}

    \raggedright \textbf{force\_shutdown}(\textit{self})

\setlength{\parskip}{2ex}
\setlength{\parskip}{1ex}
    \end{boxedminipage}

    \label{ThreadPool:ThreadPoolMixIn:serve_forever}
    \index{ThreadPool \textit{(module)}!ThreadPool.ThreadPoolMixIn \textit{(class)}!ThreadPool.ThreadPoolMixIn.serve\_forever \textit{(method)}}

    \vspace{0.5ex}

\hspace{.8\funcindent}\begin{boxedminipage}{\funcwidth}

    \raggedright \textbf{serve\_forever}(\textit{self})

    \vspace{-1.5ex}

    \rule{\textwidth}{0.5\fboxrule}
\setlength{\parskip}{2ex}
    Handle one request at a time until doomsday.

\setlength{\parskip}{1ex}
    \end{boxedminipage}

    \vspace{0.5ex}

\hspace{.8\funcindent}\begin{boxedminipage}{\funcwidth}

    \raggedright \textbf{process\_request\_thread}(\textit{self})

    \vspace{-1.5ex}

    \rule{\textwidth}{0.5\fboxrule}
\setlength{\parskip}{2ex}
    obtain request from queue instead of directly from server socket

\setlength{\parskip}{1ex}
      Overrides: SocketServer.ThreadingMixIn.process\_request\_thread

    \end{boxedminipage}

    \label{ThreadPool:ThreadPoolMixIn:handle_request}
    \index{ThreadPool \textit{(module)}!ThreadPool.ThreadPoolMixIn \textit{(class)}!ThreadPool.ThreadPoolMixIn.handle\_request \textit{(method)}}

    \vspace{0.5ex}

\hspace{.8\funcindent}\begin{boxedminipage}{\funcwidth}

    \raggedright \textbf{handle\_request}(\textit{self})

    \vspace{-1.5ex}

    \rule{\textwidth}{0.5\fboxrule}
\setlength{\parskip}{2ex}
    simply collect requests and put them on the queue for the workers.

\setlength{\parskip}{1ex}
    \end{boxedminipage}


\large{\textbf{\textit{Inherited from SocketServer.ThreadingMixIn}}}

\begin{quote}
process\_request()
\end{quote}

%%%%%%%%%%%%%%%%%%%%%%%%%%%%%%%%%%%%%%%%%%%%%%%%%%%%%%%%%%%%%%%%%%%%%%%%%%%
%%                            Class Variables                            %%
%%%%%%%%%%%%%%%%%%%%%%%%%%%%%%%%%%%%%%%%%%%%%%%%%%%%%%%%%%%%%%%%%%%%%%%%%%%

  \subsubsection{Class Variables}

    \vspace{-1cm}
\hspace{\varindent}\begin{longtable}{|p{\varnamewidth}|p{\vardescrwidth}|l}
\cline{1-2}
\cline{1-2} \centering \textbf{Name} & \centering \textbf{Description}& \\
\cline{1-2}
\endhead\cline{1-2}\multicolumn{3}{r}{\small\textit{continued on next page}}\\\endfoot\cline{1-2}
\endlastfoot\raggedright n\-u\-m\-T\-h\-r\-e\-a\-d\-s\- & \raggedright \textbf{Value:} 
{\tt 180}&\\
\cline{1-2}
\raggedright a\-l\-l\-o\-w\-\_\-r\-e\-u\-s\-e\-\_\-a\-d\-d\-r\-e\-s\-s\- & \raggedright \textbf{Value:} 
{\tt True}&\\
\cline{1-2}
\raggedright K\-E\-E\-P\-\_\-R\-U\-N\-N\-I\-N\-G\- & \raggedright \textbf{Value:} 
{\tt True}&\\
\cline{1-2}
\raggedright l\-o\-g\-g\-e\-r\- & \raggedright \textbf{Value:} 
{\tt Log()}&\\
\cline{1-2}
\raggedright d\-a\-e\-m\-o\-n\-\_\-t\-h\-r\-e\-a\-d\-s\- & \raggedright \textbf{Value:} 
{\tt True}&\\
\cline{1-2}
\end{longtable}

    \index{ThreadPool \textit{(module)}!ThreadPool.ThreadPoolMixIn \textit{(class)}|)}
    \index{ThreadPool \textit{(module)}|)}
